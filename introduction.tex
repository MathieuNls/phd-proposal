%!TEX root = research_proposal.tex

\pagenumbering{arabic}
\setcounter{page}{1}

\chapter{Introduction}

% Section that introduces the field and motivates the necessity to conduct studies in this field
Software maintenance activities such as debugging and feature enhancement are known to be challenging and costly \cite{Pressman2005}.
Studies have shown that the cost of software maintenance can reach up to 70\% of the overall cost of the software development life cycle \cite{HealthSocial2002}.
Much of this is attributable to several factors including the increase in software complexity, the lack of traceability between the various artefacts of the software development process, the lack of proper documentation,  and the unavailability of the original developers of the systems.

Research in software maintenance has evolved over the years to include areas like mining bug repositories, bug analysis, prevention and reproduction. The ultimate goal is to develop techniques and tools to help software developers detect, correct, and prevent bugs in an effective and efficient manner. Despite the recent advances in the field, the literature shows that many existing software maintenance tools have yet to be adopted by industry \cite{Lewis2013,Foss2015,Layman2007,Ayewah2007,Ayewah2008,Johnson2013,Norman2013, Hovemeyer2004, Lopez2011}. We believe that this is caused by the following factors:
\begin{itemize}
	\item Integration with the developer's workflow: Most existing maintenance tools (\cite{Kim2006a,Ayewah2008b, Findbugs2015,Moha2010,Palma,Nayrolles2013d,Nayrolles,Nayrolles2013a, Nayrolles2015a} are some noticeable examples) are not integrated well with the work flow of software developers (i.e., coding, testing, debugging, committing). Using these tools, developers have to download, install and understand them to achieve a given task. They would constantly need to switch from one workspace to another for different tasks (i.e., feature location with a command line tool, development and testing  code with an IDE, development and testing front end code with another IDE and a browser, etc.)\cite{Robertson2004,Robertson2006,Beckwith2006}.

	\item Corrective actions: The outcome of these tools does not always lead to corrective actions that the developers can implement. Most of these tools return several results that are often difficult to interpret by developers. Take for example, FindBugs \cite{Hovemeyer2004}, a popular bug detection tool. This tool detects hundreds of bug signatures and reports them using an abbreviated code such as {\tt CO\_COMPARETO\_INCORRECT\_FLOATING}. Using this code, developers can browse the FindBug's dictionary and find the corresponding definition {\it ``This method compares double or float values using pattern like this: $val1 > val2~?~1 : val1 < val2~?~-1 : 0$''. While the detection of this bug pattern is accurate, the tool does not propose any corrective actions to the developers that can help them fix the problem. Moreover, it has been reported in the literature that the output of existing maintenance tools tends to be verbose at the point where developers decide to simply ignore them \cite{Arai2014, Kim2007b, Kim2007c, Ayewah2010, Shen2011}.

	\item Leverage of historical data: These tools do not leverage a large body of knowledge that already exists in open source systems.  For defect prevention, foe example, the state of the art approaches consists of adapting statistical models built for one project to another project \cite{Lo2013, Nam2013}.	As argued by Lewis {\it et al.} \cite{Lewis2013} and Johnson {\it et al.} \cite{Johnson2013}, approaches based solely on statistical models are perceived by developers as black box solutions. Developers are less likely to trust the output of these tools.
\end{itemize}

In this thesis, we propose to address some of the above-mentioned issues by focusing on developing techniques and tools that support software maintainers at commit-time. As part of the developer's work flow, a commit marks the end of a given task or subtask  as the developer is ready to version the source code. We propose a set of approaches in which we intercept the commits and analyse them with the objective of preventing unwanted modifications to the system. By doing so, we do not only propose solutions that integrate well with the developer's work flow, but also there is no need for software developers to use any other external tools. As we will show in the rest of this proposal, some of the techniques we propose rely on best practices found in a a large repository of open source systems. In other words, we aim to leverage historical data to guide new development efforts. We refer to the field of study that encompasses software analysis techniques that operate on code commits as software maintenance at commit-time.

More precisely, we propose the following contributions that we present here and discuss in more detail in the next section:
\begin{itemize}
	\item An aggregated bug repository system.
	\item A clone prevention technique at commit-time.
	\item A bug prevention technique at commit-time
	\item A bug reproduction technique based on directed model checking and crash traces
	\item A new classification of bugs based on the locations of the corrections.
\end{itemize}


\section{Research Contributions\label{sec:objective-thesis}}

\subsection{An aggregate bug  repository for  developers  and  researchers}
% Section that motivates the need for BUMPER and the bug taxonomy
When facing a new bug, one might want to leverage decades of open source software history to find a suitable solution.
The chances are that a similar bug has already been fixed somewhere  in  another  open  source  project.
The  problem  is that each open source project hosts its data in a different data repository,  using  different  bug  tracking  and  version  control systems. Moreover,  these  systems  have  different  interfaces to  access  data.
The  data  is  not  represented  in  a  uniform way  either. This  is  further  complicated  by  the  fact  that  bug tracking tools and version control systems are not necessarily connected. The  former  follows  the  life  of  the  bug,  while  the latter  manages  the  fixes. As  a  result,  one  would  have  to search the version control system repository to find candidate solutions. Moreover,  developers  mainly  use  classical  search  engines that  index  specialized  sites  such  as  StackOverflow. These sites  are  organized  in  the  form  of  question-response  where a developer submits a problem and receives answers from the community. While the answers are often accurate and precise, they do not leverage the history of open source software that has been shown to provide useful insights to help with many maintenance activities such as bug fixing \cite{Saha2014}, bug reproduction \cite{Nayrolles2015}, fault analysis \cite{Nessa2008}, etc.

In this work, we introduce BUMPER (BUg Metarepository for  dEvelopers  and  Researchers),  a  web-based  infrastructure
that  can  be  used  by  software  developers  and  researchers  to access  data  from  diverse  repositories  using  natural  language queries in a transparent manner, regardless of where the data was originally created and hosted.
The  idea  behind  BUMPER  is  that  it  can  connect  to  any bug  tracking  and  version  control  systems  and  download  the data  into  a  single  database. We  created  a  common  schema that represents data, stored in various bug tracking and version control systems. BUMPER uses a web-based interface to allow users to search the aggregated database by expressing queries through a single point of access. This way, users can focus on the analysis itself and not on the way the data is represented or located.
BUMPER supports many features including: (1) the ability to use multiple bug tracking and control version systems, (2) the  ability  to  search  very  efficiently  large  data  repositories using both natural language and a specialized query language, (3)  the  mapping  between  the  bug  reports  and  the  fixes,  and (4)  the  ability  to  export  the  search  results  in  Json,  CSV  and XML formats.

\subsection{An incremental approach for preventing bug and clone insertion at commit time}

% Section that motivates the need for PRECINT (for preventing bug and clone insertion)
Code clones appear when developers reuse code with little to no modification to the original code.
Studies have shown that clones can account for about 7\% to 50\% of code in a given software system \cite{Baker, StephaneDucasse}. Developers often reuse code (and create clones) in their software on purpose \cite{Kim2005}.
Nevertheless, clones are considered a bad practice in software development since they can introduce new bugs in the code \cite{Kapser2006,Juergens2009,Li2006}. If a bug is discovered in one segment of the code that has been copied and pasted several times, then the developers will have to remember the places where this segment has been reused in order to fix the bug in each place. In the last two decades, there have been many studies and tools that aim at detecting clones.
They can be grouped into three categories. Although these techniques and tools have been shown to be useful in detecting clones, they operate in an off-line fashion (i.e., after the clones have been inserted). Software developers might be reluctant to use these tools on a day-today basis (i.e., as part of the continuous development process), unless they are involved in a major refactoring effort. This problem is somehow similar to the problem of adopting bug identification tools. Johnson et al. \cite{Johnson2013} showed that these tools are challenging to use because they do not integrate well with the day-to-day work flow of a developer. Also they output a large amount of data when applied to the entire system, making it hard to understand and analyse their results.

In this research, we present PRECINCT (PREventing Clones INsertion at Commit Time) that focuses on preventing the
insertion of clones at commit time, i.e., before they reach the central code repository.
PRECINCT is an online clone detection technique that relies on the use of pre-commit hooks capabilities of modern source code version control systems. A pre-commit hook is a process that one can implement to receive the latest modification to the source code done by a given developer just before the code reaches the central repository.
PRECINCT intercepts this modification and analyses its content to see whether a suspicious clone has been introduced
or not. A flag is raised if a code fragment is suspected to be a clone of an existing code segment.
In fact, PRECINCT, itself, can be seen as a pre-commit hook that detects clones that might have been inserted in the latest changes with regard to the rest of the source code.

Similar to clone detection, we propose an approach for preventing the introduction of bugs at commit-time.  Many tools exist to prevent a developer to ship {\it bad} code \cite{Dangel2000,Hovemeyer2007,Moha2010} or to identify {\it bad} code after executions (e.g in test or production environment) \cite{Nayrolles,Nayrolles2013a}. However, these tools rely on metrics and rules to statically and/or dynamically identify sub-optimum code. Our approach, called {\tt BIANCA} (Bug Insertion ANticipation by Clone Analysis at merge time), is different than the approaches presented in the literature because it mines and analyses the change patterns in commits and matches them against past commits known to have introduced a defect in the code (or that have just been replaced by better implementation).

\subsection{A bug reproduction technique based on a combination of crash traces and model checking}
When a system crashes, software developers need to reproduce the crash (usually in a lab environment) so as to provide corrective measures. A survey conducted with the developers of major open source software systems such as Apache, Mozilla and Eclipse revealed that one of the most valuable piece of information that can help locate and fix the cause of a crash is the one that can help reproduce it \cite{Bettenburg2008}. Crash reproduction is  an expensive task because the data provided by end users is often scarce \cite{Artzi2008,Jin2012,Chen2013}. It is therefore important to invest in techniques and tools for automatic bug reproduction to ease the maintenance process and accelerate the rate of bug fixes and patches.
Existing techniques can be divided into two categories: (a) On-field record and in-house replay \cite{Steven2000,Narayanasamy2005,Artzi2008,Roehm2015}, and (b) In-house crash explanation \cite{Jin2012,Jin2013,Zuddas2014,Chen2013a,Nayrolles2015}.

In this work, we propose an approach, called JCHARMING (Java CrasH Automatic Reproduction by directed Model checkING) that uses a combination of crash traces and model checking to automatically reproduce bugs that caused field failures.
Unlike existing techniques, JCHARMING does not require instrumentation of the code.
It does not need access to the content of the heap either.
Instead, JCHARMING uses a list of functions output when an uncaught exception in Java occurs (i.e., the crash trace) to guide a model checking engine to uncover the statements that caused the crash.

\subsection{A new taxonomy of bugs based on the locations of the corrections --- an empirical Study}

There have been several studies (e.g., \cite{Weiß2007, Zhang2013}) that study of the factors that influence the bug fixing time.
These studies   empirically investigate the relationship between bug report attributes (description, severity, etc.) and the fixing time. Other studies take bug analysis to another level by investigating techniques and tools for bug prediction and reproduction (e.g., \cite{Chen2013, Kim2007a, Nayrolles2015}). These studies, however, treat all bugs as the same.
For example, a bug that requires only one fix is analysed the same way as a bug that necessitates multiple fixes.
Similarly, if multiple bugs are fixed by modifying the exact same locations in the code, then we should investigate how these bugs are related in order to predict them in the future. Note here that we do not refer to duplicate bugs. Duplicate bugs are marked as duplicate (and not fixed) and only the master bug is fixed. From the bug handling perspective, if we can develop a way to detect related bug reports during triaging then we can achieve considerable time saving in the way bug reports are processed, for example, by assigning them to the same developers.We also conjecture that detecting related bugs can help with other tasks such as bug reproduction. We can  reuse the reproduction of an already fixed bug to reproduce an incoming and related bug.

We investigate the relationship between bugs by examining their locations of the fixes.
By a fix, we mean a modification (adding or deleting lines of code) to an exiting file that is used to solve the bug.We argue that bugs can be classified into four types:
A bug of Type 1 refers to a bug being fixed in one single location (i.e., one file), while Type 2 refers to bugs being fixed in more than one location.
Type 3 refers to multiple bugs that are fixed in the exact same location.
Type 4 is an extension of Type 3, where multiple bugs are resolved by modifying the same set of locations.
Note that Type 3 and Type 4 bugs are not duplicates, they may occur when different features of the system fail due to the same root causes (faults).
We conjecture that knowing the proportions of each type of bugs in a system may provide insights into the quality of the system.
Knowing, for example, that in a given system the proportion of Type 2 and 4 bugs is high may be an indication of poor system quality since many fixes are needed to address these bugs.
In addition, the existence of a high number of Types 3 and 4 bugs calls for techniques that can effectively find bug reports related to an incoming bug during triaging.
This is similar to the many studies that exist on detection of duplicates (e.g., \cite{Runeson2007, Sun2010,Nguyen2012}), except that we are not looking for duplicates but for related bugs (bugs that are due to failures of different features of the system, caused by the same faults).

\section{Outline\label{sec:outline}}

The remaining chapters of this proposal are:

\begin{itemize}
	\item Chapter \ref{chap:relwork} - {\it Background \& Related work}.
	In this chapter, we present the major studies related to our research field, namely, crash reproduction, aggregating bug repositories for mining purposes, and clone detection.

	\item Chapter \ref{chap:bumper} - {\it An Aggregate Bug Repository for Developers and Researchers}.
	In this chapter, we present {\tt BUMPER} (BUg Metarepository for  dEvelopers  and  Researchers), our bug meta-repository. {\tt BUMPER} acts as our data source for the different contributions.

	\item Chapter \ref{chap:jcharming} - {\it JCHARMING: Java CrasH Automatic Reproduction by directed Model checkING}.
	In this chapter we discuss the components of JCHARMING, the bug reproduction approach we propose.

	\item Chapter \ref{chap:bianca} - {\it Preventing Bug Insertion Using Clone Detection}. In this chapter, we present an approach named {\tt BIANCA} (Bug Insertion ANticipation by Clone Analysis at merge time) which uses clone detection to prevent bug insertion.

	\item Chapter \ref{chap:plan} - {\it Remaining Work} presents  the remaining work and a publication plan.
\end{itemize}
