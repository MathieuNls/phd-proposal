%!TEX root = research_proposal.tex

\chapter{Conclusion\label{chap:conclusion}}

The maintenance and evolution processes of system represent more than 70\% of the effort practioniers invest in them.
Hundreds of papers have been published with the aim to improve our knowledge of these processes in terms of issue triaging, issue prediction, duplicate issue detection, issue reproduction and co-changes prediction. 
All these publications gave meaning to the millions of issues that can be found in open source issue \& project and revision management systems. 
Context-aware IDE and think tank in open source architecure (\cite{chansler2011architecture}) open the path to approaches that support developpers during their programming sessions by levergaging past indexed knowledge and past architectures. 

In this research proposal, we first presented what are the most influencial papers in the different field our work lies on in Chapter \ref{chap:relwork}. 
Then, in chapter \ref{chap:thesis}, we identified current problems in the literature and our proposed solution to overcome them. 
Chapter \ref{chap:methodology} presented our proposal in details while chapter \ref{chap:plan} detailled our attempt planning.

More specifically, in Chapter \ref{chap:methodology}, we presented four approaches : {\tt BUMPER}, {\tt JCHARMING}, {\tt RESEMBLE}, {\tt BIANCA} we proposed a taxonomy of bugs. When combined into {\tt pErICOPE} \todo{Not sure yet about this name because of the Periscope app} (Ecosystem Improve source COde during Programming session with real-time mining of common knowlEdge), these tools (i) provide the possibility to search related software artifacts using natural language, (ii) accurately reproduce field-crash in lab environment, (iii) recommend improvement or completion of current block of code and (iv) prevent the introduction of clones / issues at commit time. 


{\tt BUMPER} has been designed to handle heavy traffic while {\tt JCHARMING} can reproduce 85\% of real-world issues we submitted to it. On its side {\tt BIANCA} is able to flag 41.5\% and 48.89\% of commit introducing bug as dangerous with 13.4\% and 21\% of false positive, respectively. 

Our future works, according to our publication plan described in section \ref{sec:publication-plan}, are as follows. First, we want to improve the performances of {\tt BIANCA} in terms of false positives. 
Then, create the IDE plugin that will support {\tt RESEMBLE}. Finally, we want to refined our taxonomy by including as many as datasets as possible.






